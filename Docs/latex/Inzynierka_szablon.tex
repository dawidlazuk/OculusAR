\documentclass[a4paper,11pt,twoside]{report}


% -------------- Kodowanie znaków, język polski -------------

\usepackage[utf8]{inputenc}
\usepackage[MeX]{polski}
%\usepackage[T1]{fontenc}
\usepackage[english, polish]{babel}


% ----------------- Przydatne pakiety ----------------------
\usepackage{amsfonts}
\usepackage{mathrsfs} 
\usepackage{amsmath,amsthm,latexsym,xpatch}
\usepackage[dvips]{graphicx,color}
\usepackage{enumerate}
\usepackage{enumitem}
\usepackage{verbatim}
\usepackage{array}
\usepackage{pstricks}
\usepackage{textcomp}


% ---------------- Marginesy, akapity, interlinia ------------------

\usepackage[inner=20mm, outer=20mm, bindingoffset=10mm, top=25mm, bottom=25mm]{geometry}


\linespread{1.5}
%\allowdisplaybreaks
\usepackage{indentfirst} % opcjonalnie; pierwszy akapit z wcięciem
\setlength{\parindent}{5mm}

\hyphenation{Syl-ves-tra}
\hyphenation{Syl-ves-ter-a}

%--------------------------- ŻYWA PAGINA ------------------------

\usepackage{fancyhdr}
\pagestyle{fancy}
\fancyhf{}
% numery stron: lewa do lewego, prawa do prawego 
\fancyfoot[LE,RO]{\thepage} 
% prawa pagina: zawartość \rightmark do lewego, wewnętrznego (marginesu) 
\fancyhead[LO]{\sc \nouppercase{\rightmark}}
% lewa pagina: zawartość \leftmark do prawego, wewnętrznego (marginesu) 
\fancyhead[RE]{\sc \leftmark}

\renewcommand{\chaptermark}[1]{
\markboth{\thechapter.\ #1}{}}

% kreski oddzielające paginy (górną i dolną):
\renewcommand{\headrulewidth}{0 pt} % 0 - nie ma, 0.5 - jest linia


\fancypagestyle{plain}{% to definiuje wygląd pierwszej strony nowego rozdziału - obecnie tylko numeracja
  \fancyhf{}%
  \fancyfoot[LE,RO]{\thepage}%
  
  \renewcommand{\headrulewidth}{0pt}% Line at the header invisible
  \renewcommand{\footrulewidth}{0.0pt}
}



% ---------------- Nagłówki rozdziałów ---------------------

\usepackage{titlesec}
\titleformat{\chapter}%[display]
  {\normalfont\Large \bfseries}
  {\thechapter.}{1ex}{\Large}

\titleformat{\section}
  {\normalfont\large\bfseries}
  {\thesection.}{1ex}{}
\titlespacing{\section}{0pt}{30pt}{20pt} 
%\titlespacing{\co}{akapit}{ile przed}{ile po} 
    
\titleformat{\subsection}
  {\normalfont \bfseries}
  {\thesubsection.}{1ex}{}


% ----------------------- Spis treści ---------------------------
\def\cleardoublepage{\clearpage\if@twoside
\ifodd\c@page\else\hbox{}\thispagestyle{empty}\newpage
\if@twocolumn\hbox{}\newpage\fi\fi\fi}


% kropki dla chapterów
\usepackage{etoolbox}
\makeatletter
\patchcmd{\l@chapter}
  {\hfil}
  {\leaders\hbox{\normalfont$\m@th\mkern \@dotsep mu\hbox{.}\mkern \@dotsep mu$}\hfill}
  {}{}
\makeatother

\usepackage{titletoc}
\makeatletter
\titlecontents{chapter}% <section-type>
  [0pt]% <left>
  {}% <above-code>
  {\bfseries \thecontentslabel.\quad}% <numbered-entry-format>
  {\bfseries}% <numberless-entry-format>
  {\bfseries\leaders\hbox{\normalfont$\m@th\mkern \@dotsep mu\hbox{.}\mkern \@dotsep mu$}\hfill\contentspage}% <filler-page-format>

\titlecontents{section}
  [1em]
  {}
  {\thecontentslabel.\quad}
  {}
  {\leaders\hbox{\normalfont$\m@th\mkern \@dotsep mu\hbox{.}\mkern \@dotsep mu$}\hfill\contentspage}

\titlecontents{subsection}
  [2em]
  {}
  {\thecontentslabel.\quad}
  {}
  {\leaders\hbox{\normalfont$\m@th\mkern \@dotsep mu\hbox{.}\mkern \@dotsep mu$}\hfill\contentspage}
\makeatother



% ---------------------- Spisy tabel i obrazków ----------------------

\renewcommand*{\thetable}{\arabic{chapter}.\arabic{table}}
\renewcommand*{\thefigure}{\arabic{chapter}.\arabic{figure}}
%\let\c@table\c@figure % jeśli włączone, numeruje tabele i obrazki razem


% --------------------- Definicje, twierdzenia etc. ---------------


\makeatletter
\newtheoremstyle{definition}%    % Name
{3ex}%                          % Space above
{3ex}%                          % Space below
{\upshape}%                      % Body font
{}%                              % Indent amount
{\bfseries}%                     % Theorem head font
{.}%                             % Punctuation after theorem head
{.5em}%                            % Space after theorem head, ' ', or \newline
{\thmname{#1}\thmnumber{ #2}\thmnote{ (#3)}}%  % Theorem head spec (can be left empty, meaning `normal')
\makeatother

% ----------------------------- POLSKI --------------------------------
\theoremstyle{definition}
\newtheorem{theorem}{Twierdzenie}[chapter]
\newtheorem{lemma}[theorem]{Lemat}
\newtheorem{example}[theorem]{Przykład}
\newtheorem{proposition}[theorem]{Stwierdzenie}
\newtheorem{corollary}[theorem]{Wniosek}
\newtheorem{definition}[theorem]{Definicja}
\newtheorem{remark}[theorem]{Uwaga}

% If in English, comment this and uncomment below:

% ----------------------------- ENGLISH -----------------------------
%\theoremstyle{definition}
%\newtheorem{theorem}{Theorem}[chapter]
%\newtheorem{lemma}[theorem]{Lemma}
%\newtheorem{example}[theorem]{Example}
%\newtheorem{proposition}[theorem]{Proposition}
%\newtheorem{corollary}[theorem]{Corollary}
%\newtheorem{definition}[theorem]{Definition}
%\newtheorem{remark}[theorem]{Remark}



% ----------------------------- Dowód -----------------------------

\makeatletter
\renewenvironment{proof}[1][\proofname]
{\par
  \vspace{-12pt}% remove the space after the theorem
  \pushQED{\qed}%
  \normalfont
  \topsep0pt \partopsep0pt % no space before
  \trivlist
  \item[\hskip\labelsep
        \sc
    #1\@addpunct{:}]\ignorespaces
}
{%
  \popQED\endtrivlist\@endpefalse
  \addvspace{20pt} % some space after
}

\renewcommand{\qedhere}{\hfill \qedsymbol}
\makeatother





% -------------------------- POCZĄTEK --------------------------


% --------------------- Ustawienia użytkownika ------------------

\newcommand{\tytul}{Gogle AR jako modyfikacja gogli VR}
\renewcommand{\title}{AR googles as modified VR googles}
\renewcommand{\author}{Dawid Łazuk, Piotr Piwowarski}
\newcommand{\album}{268791,000000}
\newcommand{\type}{inżyniers} % magisters, licencjac (Master or Engineer in English)
\newcommand{\supervisor}{dr inż. Krzysztof Kaczmarski}



\begin{document}
\sloppy

% \selectlanguage{english} % uncomment this for English 

% ----------------------- Abstrakty -----------------------------

%\selectlanguage{polish}
\begin{abstract}

\begin{center}
\tytul
\end{center}

Streszczam.

Lorem ipsum dolor sit amet, consetetur sadipscing elitr, sed diam nonumyeirmod tempor invidunt ut labore et dolore magna aliquyam erat, sed diamvoluptua. At vero eos et accusam et justo duo dolores et ea rebum. Stet clita kasd gubergren, no sea takimata sanctus est Lorem ipsum dolor sit amet.\\

\noindent \textbf{Słowa kluczowe:} slowo1, slowo2, ...
\end{abstract}

\null\thispagestyle{empty}\newpage

{\selectlanguage{english}
\begin{abstract}

\begin{center}
\title
\end{center}

Lorem ipsum dolor sit amet, consetetur sadipscing elitr, sed diam nonumyeirmod tempor invidunt ut labore et dolore magna aliquyam erat, sed diamvoluptua. At vero eos et accusam et justo duo dolores et ea rebum. Stet clita kasd gubergren, no sea takimata sanctus est Lorem ipsum dolor sit amet.

Lorem ipsum dolor sit amet, consetetur sadipscing elitr, sed diam nonumyeirmod tempor invidunt ut labore et dolore magna aliquyam erat, sed diamvoluptua. At vero eos et accusam et justo duo dolores et ea rebum. Stet clita kasd gubergren, no sea takimata sanctus est Lorem ipsum dolor sit amet.\\

\noindent \textbf{Keywords:} keyword1, keyword2, ...
\end{abstract}}


\null\thispagestyle{empty}\newpage

\null \hfill Warszawa, dnia ..................\\

\par\vspace{5cm}

\begin{center}
Oświadczenie % Declaration - for English
\end{center}

\indent Oświadczam, że pracę \type ką pod
tytułem ,,\tytul '', której promotorem jest \supervisor \ wykonałam/em
samodzielnie, co poświadczam własnoręcznym podpisem.
\vspace{2cm}

% English:

%I hereby declare that the thesis entitled ,,\title '', submitted for the \type ~degree, supervised  by \supervisor , is entirely my original work apart from the recognized reference.
%\vspace{2cm}

\begin{flushright}
  \begin{minipage}{50mm}
    \begin{center}
      ..............................................

    \end{center}
  \end{minipage}
\end{flushright}

\thispagestyle{empty}
\newpage

\null\thispagestyle{empty}\newpage
% ------------------- 4. Spis treści ---------------------
% \selectlanguage{english} - for English

\tableofcontents
\thispagestyle{empty}
\newpage
% -------------- 5. ZASADNICZA CZĘŚĆ PRACY --------------------
\null\thispagestyle{empty}\newpage
\setcounter{page}{11}
\pagestyle{fancy}


\chapter*{Wstęp} % Intruduction
\markboth{}{Wstęp}
\addcontentsline{toc}{chapter}{Wstęp}

Szybki rozwój techniki spowodował zwiększenie jej obecności w codziennym życiu. Rozwój interfejsów użytkownika i badania nad jego wrażeniami z wykorzystywania różnych systemów spowodowały powstanie technologii takich jak wirtualna i rozszerzona rzeczywistość. Służą one zarówno do rozrywki jak i też do ułatwienia życia i usprawnienia pracy ludziom. Rozwiązania tego typu stają się coraz popularniejsze i można zaobserwować rosnące nimi zainteresowanie w wielu branżach.

Dlatego też, w niniejszej pracy podjęto się zbadania możliwości zastosowania gogli wirtualnej rzeczywistości w aplikacjach wykorzystujacych rzeczywistość rozszerzoną.

\chapter{Wstęp teoretyczny}

W tym rozdziale wprowadzimy czytelnika w zagadnienia wirtualnej oraz rozszerzonej rzeczywistości. Omówimy też naturę wizji stereoskopowej człowieka w zakresie zjawisk, które będą miały wpływ na przebieg prac i ich rezultat.

\section {Rodzaje rzeczywistości}

\begin{definition}[Wirtualna rzeczywistość]
\textit{Wirtualną rzeczywistością} (ang. VR - Virtual Reality) nazywamy wrażenie przebywania użytkownika w świecie stworzonym wirtualnie. Zwykle dotyczy to obrazu, dźwięku oraz interakcji z otoczeniem, ale podejmowane są próby wykorzystania zapachu i dotyku w aplikacjach VR.
\end{definition}

\begin{definition}[Rozszerzona rzeczywistość]
\textit{Rozszerzoną rzeczywistością} (ang. AR - Augmented Reality) nazywamy nałoźenie na wizję świata rzeczywistego obrazu stworzonego wirtualnie. Użytkownik dzięki temu może na przykład otrzymywać dodatkowe informacje o obiektach, na które patrzy. Systemy AR nie zapewniają interakcji z częścią wirtualną świata.
\end{definition}

\begin{definition}[Mieszana rzeczywistość]
\textit{Mieszaną rzeczywistością} (ang. MR - Mixed Reality) nazywamy rozszerzenie klasy rozwiązań AR o interakcję użytkownika z wirtualną częścią świata.
\end{definition}

W niniejszej pracy zajmiemy się rozwiązaniami wirtualnej i rozszerzonej rzeczywistości. Definicja rzeczywistości mieszanej została zamieszczona w celu rozróżnienia jej od rozszerzonej, jednak to zagadnienie leży poza zakresem pracy dyplomowej.

\section {Wizja stereoskopowa człowieka}

Człowiek doświadcza widzenia trójwymiarowego dzięki temu, że dwoje jego oczu jest skierowane mniej więcej równolegle w ten sam punkt. Umożliwia to zebranie dwóch obrazów, które różnią się od siebie perspektywą obserwatora. Mózg przetwarzając te obrazy tworzy efekt głębi umożliwiając człowiekowi ocenę odległości przedmiotów, na które patrzy.

\subsection {Odległość między źrenicami}

Badania antropometryczne personelu armii Stanów Zjednoczonych w 2012 roku objęły swoim zakresem odległości między ludzkimi źrenicami. Wyniki tych badań są zawarte w poniższej tabeli. Dane są przedstawione w milimetrach.

\begin{table}[h!]
\centering
\label{my-label}
\begin{tabular}{lcccc}
\textbf{Płeć} & \multicolumn{1}{l}{\textbf{Minimum}} & \multicolumn{1}{l}{\textbf{Maksimum}} & \multicolumn{1}{l}{\textbf{Średnia}} & \multicolumn{1}{l}{\textbf{Mediana}} \\
Kobieta       & 51.0                                 & 74.5                                  & 61.7                                 & 62.0                                 \\
Mężczyzna     & 53.0                                 & 77.0                                  & 64.0                                 & 64.0                                
\end{tabular}
\caption{Odległości między źrenicami}
\end{table}

Z powyższych danych jednoznacznie wynika, że odległość pomiędzy ludzkimi źrenicami zawiera się w przedziale od 51.0 milimetrów do 77.0 milimetrów.

\textbf{(http://www.dtic.mil/docs/citations/ADA634277 - do przypisów)}

\subsection {Akomodacja oka}

Ludzkie oko dostosowuje się do odległości od oglądanego przedmiotu w celu zapewnienia ostrości obrazu. Kiedy człowiek zmienia punkt skupienia wzroku na obiekt położony w innej odległości niż poprzedni, kształt jego soczewki ulega zmianie. Tym samym zmienia się też jego ogniskowa. 

Zakres akomodacji oka wynosi od 10 centymetrów (punkt bliży wzrokowej) do 6 metrów (punkt dali wzrokowej). Poza tym zakresem akomodacja nie wpływa na ostrość obrazu.

\subsection {Zbieżność oczu}

W sytuacji, kiedy człowiek patrzy na przedmiot położony daleko od niego, osie widzenia każdego z oczu są położone równolegle do siebie. Jednak kiedy człowiek obserwuje obiekt położony blisko, osie widzenia przecinają się w obserwowanym punkcie. To zjawisko, obrócenia oczu do środka, nazywane jest zbieżnością.

\begin{figure}[h]
\centering
\includegraphics[scale=0.5]{zbieznosc}
\caption[Zbieżność oczu (źródło: http://www.swiatlo.tak.pl/1/index.php/funkcje-wzroku-akomodacja-adaptacja-zbieznosc/)]{Zbieżność oczu}
\end{figure}

\chapter{Cel i założenia projektu} % Cel i założenia projektu

Celem pracy jest stworzenie systemu umożliwiającego przekazanie obrazu z dwóch kamer do gogli wirtualnej rzeczywistości), tworząc tym samym system rozszerzonej rzeczywistości. 

System ma umożliwiać efekt widzenia stereoskopowego, aby użytkownik miał wrażenie normalnej wizji. Efekt ten zostanie osiągnięty poprzez umiejscowienie kamer tak, aby odległość między ich soczewkami odpowiadała odległości pomiędzy źrenicami ludzkich oczu. 

System ma umożliwiać przetwarzanie przechwyconego obrazu zanim trafi on do oczu. Przetwarzanie to może być realizowane na przykład poprzez nałożenie prostych filtrów na obraz, lub też wykorzystania bardziej zaawansowanych algorytmów (na przykład wykrywania twarzy).


\section{Środowisko sprzętowe}

W pracy wykorzystaliśmy posiadane przez wydział MiNI gogle wirtualnej rzeczywistości Oculus Rift oraz dwie kamery USB Logitech Webcam C910 Pro HD.

\subsection{Gogle Oculus Rift}

Oculus Rift to gogle wirtualnej rzeczywistości stworzone przez firmę Oculus VR, będącą obecnie własnością Facebook Inc. Zostały ujawnione światu w 2012 roku na platformie Kickstarter w celu zebrania funduszy na rozwój projektu. 

Oculus Rift wykorzystuje dwa ekrany OLED, po jednym na każde oko. Ekrany te mają rozdzielczość 1080 pikseli na 1200 pikseli. Częstotliwość ich odświeżania wynosi 90 Hz. Pomiędzy ekranami, a oczami użytkownika znajdują się soczewki. Kąty widzenia gogli to 90\textdegree  pionowo i 80\textdegree  poziomo.

-- tutaj można wstawić jakiś schemat oculusa --

\subsection{Logitech Webcam C910 Pro HD}

C910 Pro HD to kamera USB stworzona przez firmę Logitech. Umożliwia przechwytywanie obrazu o rozdzielczości 1920 pikseli na 1080 pikseli. Posiada wbudowany mikrofon, jednak nie będzie on wykorzystywany w naszej pracy.

W związku z tym, że panoramiczny obraz nie pokrywa się z naturą ludzkiego oka ograniczymy się do obrazu 4:3. Zatem rozdzielczość obrazu w naszej pracy będzie wynosiła 1440 pikseli na 1080 pikseli na jedno oko.

Obudowa tej kamery jest dość duża, szczególnie w pozycji poziomej. Jest to problematyczne, ponieważ minimalna odległość między soczewkami, na jaką możemy umieścić kamery w tej pozycji, wynosi około 100 milimetrów. Jest to znacznie powyżej górnego ograniczenia odległości między ludzkimi źrenicami. W związku z tym konieczne będzie umieszczenie tych kamer w pozycji pionowej. Skutkuje to tym, że efektywna rozdzielczość przechwytywanego obrazu będzie wynosiła 1080 pikseli na 1440 pikseli.


\section{Środowisko programowe}

\subsection{.NET Framework}

System zostanie wykonany z wykorzystaniem .NET Framework. Językiem programowania będzie C\#. Zdecydowaliśmy się na tą technologię, ponieważ mamy w niej największe doświadczenie. Również silnik Unity3D, który wykorzystaliśmy wspiera .NET co sugerowało nam uniknięcie przyszlych problemów integracją modułów. W pracy .NET Framework zostanie wykorzystany do napisania funkcjonalności pobierania obrazu z kamer i przetwarzania go, zanim trafi do modułu przekazywania obrazu do gogli.

\subsection{EmguCV}

Przechwytywanie obrazu z kamer oraz przetwarzanie go zostało zaimplementowane z wykorzystaniem biblioteki EmguCV, która opakowuje popularną bibliotekę OpenCV w interfejs umożliwiający wykorzystanie jej w aplikacjach napisanych w języku C\#.

\subsection{Unity3D}

-- Piotrek ty tutaj opisz co nie co bo wiesz to lepiej

\section {Schemat systemu}

\subsection {Architektura aplikacji}

-- schemat, opis modułów i wykorzystane technologie

\subsection {Droga obrazu}

-- też może schemat (chyba w prezentacji na seminarium był), opis słowny

\chapter {Implementacja}

-- Tutaj wrzucimy szczegółowy techniczny opis modułów. Jako szczegółowy opis mam na myśli jego kluczowe funkcjonalności, jak wielowątkowe pobieranie obrazu itp.


\begin{definition}[Definicja]
\textit{Równaniem} nazywamy formę zdaniową postaci $t_1 = t_2$, gdzie $t_1, t_2$ są termami przynajmniej jeden z nich zawiera pewną zmienną.
\end{definition}

\begin{example}
Przykładem równania jest
\begin{equation}
2+2=4.
\end{equation}
Jeśli nie chcemy numerka, piszemy
\begin{equation*}
2+2=4.
\end{equation*}
Równanie (\ref{rownanie}) jest fałszywe. Referencje (i kilka innych rzeczy) działają po dwukrotnym przekompilowaniu tex-a.
\begin{equation}\label{rownanie}
\int \limits_{0}^{1} x \; dx = \frac{3}{2}.
\end{equation}

\end{example}

Twierdzenie \ref{Pitagoras} jest bardzo ciekawe.

\begin{theorem}[Twierdzenie Pitagorasa]\label{Pitagoras}
Niech będzie dany trójkąt prostokątny o przyprostokątnych długośći $a$ i $b$ oraz przeciwprostokątnej długości $c$. Wtedy
$$
a^2 + b^2 = c^2.
$$
\end{theorem}

\begin{proof}
Dowód został zaprezentowany w \cite{Ktos} oraz \cite{Innyktos}. Czyli w sumie mogę napisać, że w \cite{Ktos, Innyktos}. Albo że łatwo widać.
\end{proof}

\begin{corollary}
Doszedłem do jakiegoś wniosku i daję temu wyraz.
\end{corollary}




\begin{remark}
Lorem ipsum dolor sit amet, consetetur sadipscing elitr, sed diam nonumyeirmod tempor invidunt ut labore et dolore magna aliquyam erat, sed diamvoluptua. At vero eos et accusam et justo duo dolores et ea rebum.
\end{remark}

\begin{lemma}[Lemacik]
Ten lemat jest nie na temat.
\end{lemma}
\begin{proof} Dowód przez indukcję.
\end{proof}


Lorem ipsum dolor sit amet, consetetur sadipscing elitr, sed diam nonumyeirmod tempor invidunt ut labore et dolore magna aliquyam erat, sed diamvoluptua. At vero eos et accusam et justo duo dolores et ea rebum. Stet clita kasd gubergren, no sea takimata sanctus est Lorem ipsum dolor sit amet.Lorem ipsum dolor sit amet, consetetur sadipscing elitr, sed diam nonumyeirmod tempor invidunt ut labore et dolore magna aliquyam erat, sed diamvoluptua. At vero eos et accusam et justo duo dolores et ea rebum. Stet clita kasd gubergren, no sea takimata sanctus est Lorem ipsum dolor sit amet.



\section{Tabele i rysunki}

 Opcjonalny argument środowisk table i figure\\
h 	- 	bez przemieszczenia, dokładnie w miejscu użycia (uzyteczne w odniesieniu do niewielkich wstawek); \\
t 	- 	na górze strony;\\
b 	- 	na dole strony;\\
p 	- 	na stronie zawierającej wyłącznie wstawki;\\
! 	- 	ignorując większość parametrów kontrolujacych umieszczanie wstawek, przekroczenie wartosci, których może nie pozwolić na umieszczanie nastepnych wstawek na stronie.

\begin{figure}[h!]
\begin{center}
\setlength{\unitlength}{1mm}
\begin{picture}(40, 30)
\put(20,1){\line(0,1){20}} % linia

% dół
\put(20,1){\circle*{2}}
\put(25,1){0}

% góra
\put(20,21){\circle*{2}}
\put(25,21){1}
\end{picture}
\end{center}
\caption{Obrazek zrobiony w LaTeXu}
\end{figure}

\begin{table}[h!]
\centering
\begin{tabular}{rl|c}
bla & blabla & blablabla\\
\hline
bla & blabal & blablabla \\
ble & bleble & blebleble
\end{tabular}
\caption[Opis skrócony]{Pełny opis znajdujący się pod tabelą}
\end{table}

Lorem ipsum dolor sit amet, consetetur sadipscing elitr, sed diam nonumyeirmod tempor invidunt ut labore et dolore magna aliquyam erat, sed diamvoluptua. At vero eos et accusam et justo duo dolores et ea rebum. Stet clita kasd gubergren, no sea takimata sanctus est Lorem ipsum dolor sit amet.Lorem ipsum dolor sit amet, consetetur sadipscing elitr, sed diam nonumyeirmod tempor invidunt ut labore et dolore magna aliquyam erat, sed diamvoluptua. At vero eos et accusam et justo duo dolores et ea rebum. Stet clita kasd gubergren, no sea takimata sanctus est Lorem ipsum dolor sit amet.

\begin{figure}[h!]
\centering
\includegraphics[scale=0.5]{politechnika}
\caption[Logo MiNI]{Jakiś obrazek}
\end{figure}


\chapter{Następny rozdział}



\section{Jakiś podrozdział}


\begin{definition}
Niech $A\neq \emptyset$, $n \in \mathbb{N}$. Każde przekształcenie $f:A^n \rightarrow A$ nazywamy \textit{$n$-arną operacją} lub \textit{działaniem} określonym na $A$.
0-arne operacje to wyróżnione stałe.
\end{definition}


\begin{definition}[Algebra]
Parę uporządkowaną $(A,F)$, gdzie $A\neq \emptyset$ jest zbiorem, a $F$ jest rodziną operacji określonych na $A$, nazywamy \textit{algebrą} (lub \textit{$F$-algebrą}). Zbiór $A$ nazywa się \textit{zbiorem elementów}, \textit{nośnikiem} lub \textit{uniwersum} algebry $(A,F)$, a $F$ \textit{zbiorem operacji elementarnych}.
\end{definition}

\begin{proposition}
Stwierdzam więc ostatnio, że doszedłszy do granicy, pozostaje mi tylko przy tej granicy biwakować albo zawrócić, możliwie też szukać przejścia czy wyjścia na nowe obszary.
\end{proposition}



% -------------------- 6. Bibliografia -----------------------
% Bibliografia leksykograficznie wg nazwisk autorów

\begin{thebibliography}{20}%jak ktoś ma więcej książek, to niech wpisze większą liczbę
% \bibitem[numerek]{referencja} Autor, \emph{Tytuł}, Wydawnictwo, rok, strony
% cytowanie: \cite{referencja1, referencja 2,...}

\bibitem[1]{Ktos} A. Aaaaa, \emph{Tytuł}, Wydawnictwo, rok, strona-strona.
\bibitem[2]{Innyktos} J. Bobkowski, S. Dobkowski, \emph{Blebleble}, Magazyn nr, rok, strony.
\bibitem[3]{B} C. Brink, \emph{Power structures}, Algebra Universalis 30(2), 1993, 177-216.
\bibitem[4]{H} F. Burris, H. P. Sankappanavar, \emph{A Course of Universal Algebra}, Springer-Verlag, New York, 1981.
\end{thebibliography}
\thispagestyle{empty}


% --- 7. Wykaz symboli i skrótów - jeśli nie ma, zakomentować
\chapter*{Wykaz symboli i skrótów}

\begin{tabular}{cl}
nzw. & nadzwyczajny \\
* & operator gwiazdka \\
$\widetilde{}$ & tylda
\end{tabular}
\thispagestyle{empty}


% ----- 8. Spis rysunków - jeśli nie ma, zakomentować --------
\listoffigures
\thispagestyle{empty}


% ------------ 9. Spis tabel - jak wyżej ------------------
\renewcommand{\listtablename}{Spis tabel}
\listoftables
\thispagestyle{empty}



% 10. Spis załączników - jak nie ma załączników, to zakomentować lub usunąć

\chapter*{Spis załączników}
\begin{enumerate}[itemsep = 0pt]
\item Załącznik 1
\item Załącznik 2
\end{enumerate}
\thispagestyle{empty}

% --------------------- 11. Załączniki ---------------------
% to jest po to, żeby było wiadomo, że załączniki znajdują się na końcu pracy

\newpage
\pagestyle{empty} 
Załącznik 1, załącznik 2
\end{document}
